\section{Specifications}
\subsection{Machine Block Diagram}


\tikzstyle{block} = [draw, rectangle, minimum height=4.5em, text width=5em, text centered, font=\small]
\tikzstyle{arrow} = [thick,-{Stealth[length=3mm,width=2mm]}]

\begin{figure}[!h]
    \begin{tikzpicture}[node distance=2cm, every node/.style={inner sep=0.1cm, outer sep=0}]
      \node [block, inner sep=0cm] (clothsource) {Cloth Source};
      \node [block, right=1cm of clothsource] (solventchamber) {Solvent Chamber};
      \node [block, right=1cm of solventchamber] (pressurejet) {Pressure Jet Cleaning};
      \node [block, right=1cm of pressurejet] (thermaldrying) {Thermal Drying\\Chamber};
      \node [block, right=1cm of thermaldrying] (clothcutting) {Cloth Cutting\\Chamber};
    
      \draw [arrow] (clothsource) -- (solventchamber);
      \draw [arrow] (solventchamber) -- (pressurejet);
      \draw [arrow] (pressurejet) -- (thermaldrying);
      \draw [arrow] (thermaldrying) -- (clothcutting);
    
      \node[coordinate, right=0.7cm of clothcutting] (outputcoord) {};
      \draw [arrow] (clothcutting) -- (outputcoord);
      \node[above=0.1cm, right=0.1cm, text width=3em, text centered] at (outputcoord) {Cleaned Cloth};
    \end{tikzpicture}
      \caption{Block Diagram}
      \label{fig:block_diagram}
\end{figure}
\vspace{-1.5em}
\subsection{Solvent Chamber} 
\subsubsection{Energy Specifications}
\begin{itemize}
    \item[$\scriptstyle\circ$] \SI{220}{\volt} High Pressure Pump.
\end{itemize}
\subsubsection{Space Specifications}


\begin{itemize}
    \item[$\scriptstyle\circ$] Solvent tank - \SI{0.5}{\meter}$\times$\SI{2.1}{\meter}$\times$\SI{2.2}{\meter}.
    \item[$\scriptstyle\circ$] detergent height - \SI{0.1}{\meter}
    \item[$\scriptstyle\circ$] Volume - \SI{105}{\liter} of detergent to be used.
    
\end{itemize}
\subsubsection{Power Specifications}
\begin{itemize}
    \item[$\scriptstyle\circ$] Power required for each pump: \SI{2.7}{\kilo\watt}.
\end{itemize}
\subsubsection{Cost Specifications}
\begin{itemize}
    \item[$\scriptstyle\circ$] A waterproof container.
    \item[$\scriptstyle\circ$] 7$\times$pulleys .
    \item[$\scriptstyle\circ$] 2$\times$Nylon Brushes .
    \item[$\scriptstyle\circ$] Detergent Powder at Rs\(\,150/\text{kg}\).
    
    
\end{itemize}

\subsubsection{Performance Specifications}

\begin{itemize}
    \item[$\scriptstyle\circ$] We have assumed standard conditions with a \SI{52}{\meter} cloth(\SI{11}{\kilogram} weight) being washed in \SI{40}{\minute}.
     \item[$\scriptstyle\circ$] At any given time, \SI{52}{\meter} of cloth is in the machine. 
     \item[$\scriptstyle\circ$] Considering that \SI{13}{\meter} will be wet due to detergent, resulting in a \SI{0.5}{\kilogram} weight increase for each meter, the total weight addition is \SI{6.5}{\kilogram}. \cite{kane_evaluation_2012} \cite{eduok_effect_2021}


\end{itemize}
\subsubsection{Man Power Specifications}

\subsubsection{Milestone Specifications}
\subsection{Pressure Washing Chamber}
\subsubsection{Energy Specifications}
\begin{itemize}
    \item[$\scriptstyle\circ$] \SI{220}{\volt} High Pressure Pump.
\end{itemize}
\subsubsection{Space Specifications}
\begin{itemize}
    \item[$\scriptstyle\circ$] Enclosure made of transparent material like fibreglass(not necessarily water-tight) with dimensions: \SI{1}{\meter}$\times$\SI{2}{\meter}.
\item[$\scriptstyle\circ$] Tiles for supporting clothes during pressure washing: 2$\times$(50 \unit{\centi\metre}$\times$\SI{2}{\meter}) (preferably slippery material)
    \item[$\scriptstyle\circ$] Pump dimensions: 30 \unit{\centi\metre} $\times$15 \unit{\centi\metre}$\times$10 \unit{\centi\metre}
    \item[$\scriptstyle\circ$] Pipe dimensions: 0.25 inches(internal) - 1 inch(external) diameter
    \item[$\scriptstyle\circ$] Nozzle size: 2-5 inches
\end{itemize}
\subsubsection{Power Specifications}
\begin{itemize}
    \item[$\scriptstyle\circ$] Power required for each pump: \SI{2.7}{\kilo\watt}.
\end{itemize}

\subsubsection{Cost Specifications}
\begin{itemize}
    \item[$\scriptstyle\circ$] 2$\times$Pumps = Rs 6000$\times$2 = Rs 12000
    \item[$\scriptstyle\circ$] 10$\times$Nozzles
    \item[$\scriptstyle\circ$] 8$\times$221 Connectors(Threaded Tees)
    \item[$\scriptstyle\circ$] Pipes at Rs 150/m: Exact cost to be decided after confirming the length required.
\end{itemize}

\subsubsection{Performance Specifications}
\begin{itemize}
    \item[$\scriptstyle\circ$] Pressure rating for the pump: 250-270 bars.
    \item[$\scriptstyle\circ$] Pipes made of \Gls{Polypropylene Random Copolymer} with the working pressure of 250 bars.
    \item[$\scriptstyle\circ$] Flat fan nozzles with spray angle in the range $75^\circ$-$110^\circ$ and pressure rating of 50 bars.(Attachable to pipes of given dimensions)
    \item[$\scriptstyle\circ$] Pressure rating for the connectors: 250 bars \cite{noauthor_thermoplastic_nodate}
\end{itemize}

\subsubsection{Man Power Specifications}

\subsubsection{Milestone Specifications}
\subsection{Drying Chamber}
\subsubsection{Energy Specifications}
\begin{itemize}
    \item[$\scriptstyle\circ$]\SI{220}{\volt}, 50 Hz 3 phase for the Industrial Burner.
    \item[$\scriptstyle\circ$] \SI{220}{\volt} Industrial Exhaust Fan.
\end{itemize}
\subsubsection{Space Specifications}
\begin{itemize}
    \item[$\scriptstyle\circ$]Thermal Chamber Dimensions: \SI{3}{\meter}$\times$\SI{3}{\meter}$\times$\SI{3}{\meter}.
    \item[$\scriptstyle\circ$]Industrial Grade Burner Dimensions: 20 inch$\times$ 14 inch$\times$ 18 inch.
\end{itemize}

\subsubsection{Power Specifications}
\begin{itemize}
    \item[$\scriptstyle\circ$]Burner requires \SI{300}{\kilo\watt} for a three phase burner
    \item[$\scriptstyle\circ$]Exhaust Fan requiring \SIrange{3}{9}{\kilo\watt}, may be switched off at times
\end{itemize}

\subsubsection{Cost Specifications}
\begin{itemize}
    \item[$\scriptstyle\circ$]2$\times$Rollers/pulleys for mechanical squeezing =
    \item[$\scriptstyle\circ$]5$\times$Rollers for the thermal chamber =
    \item[$\scriptstyle\circ$]2$\times$Burner = Rs $5500\times2$ = Rs 11,000. 
    \item[$\scriptstyle\circ$]Exhaust Fan = Rs 35000.
    \item[$\scriptstyle\circ$]Sensor Temperature and Humidity :- Rs $2000 + 2000$ = Rs 4000.
    \item[$\scriptstyle\circ$]	Chamber made out of Scrap Metal :- Rs 70000 at Rs \(36/\text{kg}\).
    \item[$\scriptstyle\circ$] Aluminium Insulation sheet for the Scrap Metal :- Rs 3600
    \item[$\scriptstyle\circ$] \Gls{Polyurethane Laminate} padding for the Mechanical Squeezers :- Rs 500 at Rs \(80/\text{meter}\)
\end{itemize}


\subsubsection{Performance Specifications}
\begin{itemize}
    \item[$\scriptstyle\circ$]	2 Rollers padded with Polyurethane Laminate to squeeze out the excess water and push the cloth forward to the thermal chamber.
    \item[$\scriptstyle\circ$] Maintaining the Thermal Chamber at \SI{145}{\degreeCelsius} ideally and realistically \SI{100}{\degreeCelsius}
    \item[$\scriptstyle\circ$] Passing the cloth in 5 layers over the burner using pulleys and rollers.
    \item[$\scriptstyle\circ$] For Proper drying we will be keeping the cloth in the thermal chamber for around 10 minutes.
    \item[$\scriptstyle\circ$] Using Temperature Sensor and Humidity Sensor to monitor the inside of the chamber and further regulate the conditions using an exhaust fan. \cite{dryden_chapter_1982} \cite{noauthor_drying_nodate} \cite{noauthor_dry_nodate}
\end{itemize}
\subsubsection{Man Power Specifications}

\subsubsection{Milestone Specifications}

\subsection{Cutting Section}
\subsubsection{Energy Specifications}
\begin{itemize}
    \item \SI{220}{\volt}, 60 Hz AC for the Handheld cutter, rated at \SI{250}{\watt}, \SI{0.8}{\ampere}.
\end{itemize}
\subsubsection{Space Specifications}
\begin{itemize}
    \item[$\scriptstyle\circ$] Clamp sizing specifications: 15 \unit{\centi\metre}, and opens out to a width of 10 \unit{\centi\metre}
    \item[$\scriptstyle\circ$] Supporting table kept at an angle of \SI{70}{\degree}, slightly wider than the width of the cloth, leading the cut cloth to a pile: Dimensions to be decided according to later design specifications. 
    \item[$\scriptstyle\circ$] Handheld cutting tool specifications: Housing 19.1 \unit{\centi\metre}$\times$10.2 \unit{\centi\metre}$\times$26.2 \unit{\centi\metre}, with a circular/octagonal blade of \SI{100}{\milli\meter} width.
\end{itemize}
\subsubsection{Power Specifications}
\begin{itemize}
    \item[$\scriptstyle\circ$] The circular hand-held cutter(to be integrated into an automated mechanism) operates at \SI{250}{\watt}.
    \item[$\scriptstyle\circ$] The clamp and cutter are operated using a synchronization circuit, which uses negligible power.
\end{itemize}
\subsubsection{Cost Specifications}
\begin{itemize}
    \item[$\scriptstyle\circ$] Rs 5000 - Rs 10,000 for the cutter
    \item[$\scriptstyle\circ$] 4 Clamps can be fabricated at nominal cost, operating similar to (larger) cloth drying pins. 
\end{itemize}
\subsubsection{Performance Specifications}
\begin{itemize}
    \item[$\scriptstyle\circ$] Cutting Capacity of Blade: Upto \SI{27}{\milli\meter} thickness. 
    \item[$\scriptstyle\circ$] Blade rotates at 1000-3500 RPM. \cite{noauthor_fabric_nodate} \cite{noauthor_yj-d108_nodate}
\end{itemize}
\subsubsection{Man Power Specifications}

\subsubsection{Milestone Specifications}